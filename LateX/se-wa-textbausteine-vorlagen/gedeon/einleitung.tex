\chapter{Einleitung}\pagenumbering{arabic}
\textit{\glqq Leider ist dieses Video in Deutschland nicht verfügbar, da es Musik enthalten könnte, für die die GEMA die erforderlichen Musikrechte nicht eingeräumt hat. Das tut uns leid\grqq}

Millionen von deutschen Nutzern müssen täglich diese Sperrtafel in Kauf nehmen, wenn sie versuchen, lizenzpflichtige Inhalte auf YouTube anzuschauen, für die die Urheber nicht angemessen entlohnt wurden. Seit Jahren streiten die \gls{gema} und YouTube über diese Rechtslage. In der folgenden Seminararbeit wird die Entstehungsgeschichte sowie die Vereinsstruktur der GEMA aufgezeigt, sowie deren Rechtsgrundlage, die die Entrichtung von Gebühren enthält, näher eruiert. Darauf aufbauend soll der Rechtsstreit zwischen beiden Parteien chronologisch dargestellt werden, wobei neben dem Streit um die Sperrtafeln zwei weitere bedeutende Konflikte analysiert werden. Abschließend werden Kritiken aus unterschiedlichen Perspektiven, wie der eines Mitgliedes oder des Endnutzer dargelegt.

Ziel ist es, die Rechtslage der Entrichtung von GEMA-Gebühren am Beispiel von YouTube darzustellen sowie einen generellen Überblick über die größte deutsche Verwertungsgesellschaft zu erlangen und den neuesten Stand des Rechtsstreits zu erfahren.