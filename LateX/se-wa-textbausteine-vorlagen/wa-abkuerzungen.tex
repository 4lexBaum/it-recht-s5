% 2012-03-22 Verwendung des optionalen Parameters f\"ur die Pluralform einer Abk\"urzung
%
% 2012-02-06 Umstellung auf die neuen Kommandos
%
%
%
%  J\"org Baumgart
%  Definition einiger Abk\"urzungen
%  


% Definition von Abk\"urzungen
%
% 1. Parameter: Schluessel (key) der Abkuerzung
% 2. Parameter: Abkuerzung
% 3. Parameter: Vollform
% 4. Parameter: Vollform im Plural (optional; falls nicht definiert, wird der Wert des dritten Parameters verwendet)
%
\seNewAcronymEntry{dm}{DM}{Diagonalmatrix}{Diagonalmatrizen}

\seNewAcronymEntry{dhbw}{DHBW}{Duale Hochschule Baden-W\"urttemberg}{}{}

\seNewAcronymEntry{usb}{USB}{Universal Serial Bus}{}

\seNewAcronymEntry{ctan}{CTAN}{Comprehensive \TeX{} Archive Network}{}

\seNewAcronymEntry{gema}{GEMA}{Gesellschaft für musikalische Aufführungs- und mechanische Vervielfältigungsrechte}{}

\seNewAcronymEntry{gdt}{GDT}{Genossenschaft Deutscher Tonsetzer}{}

\seNewAcronymEntry{afma}{AFMA}{Anstalt für musikalisches Aufführungsrecht}{}

\seNewAcronymEntry{ammre}{AMMRE}{Anstalt für mechanisch-musikalische Rechte GmbH}{}

\seNewAcronymEntry{olg}{OLG}{Oberlandesgericht}{}

\seNewAcronymEntry{akm}{AKM}{Österreichische Gesellschaft der Autoren, Komponisten und Musikverleger}{}

\seNewAcronymEntry{urhg}{UrhG}{Gesetz über Urheberrecht und verwandte Schutzrechte (Urheberrechtsgesetz)}{}
% 2012-03-24
% \"Uber den optionalen Parameter in eckigen Klammern wird die Pluralform f\"ur das erste 
% Auftreten der Abk\"urzung definiert

\seNewAcronymEntry[URLs]{url}{URL}{Uniform Resource Locator}%
{Uniform Resource Locators}

\seNewAcronymEntry{vgg}{VGG}{Verwertungsgesellschaftengesetz}{}

\seNewAcronymEntry{urhwg}{UrhWG}{Urheberrechtswahrnehmungsgesetz}{}


% Definition von Symbolen
%
% 1. Parameter: Schluessel (key) des Symbols
% 2. Parameter: Symbol
% 3. Parameter: Text, der die Sortierreihenfolge festlegt (optional; falls nicht definiert, wird der Wert des zweiten 
%                        Parameters verwendet)
% 4. Parameter: Beschreibung des Symbols
%

\seNewSymbolEntry{ND}{ND}{a}{Nutzungsdauer einer Maschine}

\seNewSymbolEntry{pi}{$\pi$}{b}{Die Kreiszahl}




% Definition von Glossareintraegen
%
% 1. Parameter: Schluessel (key) des Glossareintrags
% 2. Parameter: Begriff, der im Glossar definiert wird
% 3. Parameter: Pluralform des Begriffes (optional; falls nicht definiert, wird der Wert des zweiten Parameters verwendet)
%                        Achtung: Pluralform gilt nur fuer das erste Auftreten des Begriffes im Text
% 4. Parameter: Beschreibung des Glossareintrags
%
%
%

\seNewGlossaryEntry{glos:AD}{Active Directory}{Active Directories}
{Active Directory ist in einem Windows 2000/Windows
Server 2003-Netzwerk der Verzeichnisdienst, der die zentrale
Organisation und Verwaltung aller Netzwerkressourcen erlaubt. Es
erm\"oglicht den Benutzern \"uber eine einzige zentrale Anmeldung den
Zugriff auf alle Ressourcen und den Administratoren die zentral
organisierte Verwaltung, transparent von der Netzwerktopologie und
den eingesetzten Netzwerkprotokollen. Das daf\"ur ben\"otigte
Betriebssystem ist entweder Windows 2000 Server oder
Windows Server 2003, welches auf dem zentralen
Dom\"anencontroller installiert wird. Dieser h\"alt alle Daten des
Active Directory vor, wie z.\,B. Benutzernamen und
Kennw\"orter.\protect\footnote{Bedauerlicherweise wei{\ss} der Autor dieses Dokumentes nicht mehr, woher diese Information stammt -- das 
geht in einer richtigen wissenschaftlichen Arbeit nat\"urlich \"uberhaupt nicht!!!}}
%\protect\seFootcite{Vgl.}{S. 200}{Dud09}}


\seNewGlossaryEntry{glos:bs}{Betriebssystem}{Betriebssysteme}{Die Begriffsdefinition sollten Sie eigentlich kennen!}


% Definition von Glossareintraegen, die gleichzeitig im Abk�rzungsverzeichnis auftreten
%
% 1. Parameter: Schluessel (key) des Glossareintrags
% 2. Parameter: Abk\"urzung
% 3. Parameter: Vollform
% 4. Parameter: Vollform im Plural (optional; falls nicht definiert, wird der Wert des dritten Parameters verwendet)
% 5. Parameter: Beschreibung des Glossareintrags

\seNewAcronymGlossaryEntry{glos:ma}{MA}{Mobile Applikation}{Mobile Applikationen}
{Eine Applikation, die auf einem mobilen Endger\"at ausgef\"uhrt wird.}

% 2012-03-24
% \"Uber den optionalen Parameter in eckigen Klammern wird die Pluralform f\"ur die Abk\"urzung definiert

\seNewAcronymGlossaryEntry[TAen]{glos:ta}{TA}{Transaktion}%
{Transaktionen}%
{Was eine Transaktion ist, sollten Sie ebenfalls bereits wissen!}





% Alternative Definition von Abk\"urzungen; diese sollten nicht verwendet werden!!!
%
%\newacronym{dhbw}{DHBW}{Duale Hochschule Baden-W\"urttemberg}
%\newacronym{usb}{USB}{Universal Serial Bus}


% Alternative Definition von Symbolen
%
% Achtung: ohne sort wird nach Name sortiert
%\newglossaryentry{pi}{
%name=$\pi$,
%description={Die Kreiszahl},
%type=symbolslist,
%sort=b
%}
%
%\newglossaryentry{ND}{
%name=$\mbox{\textsl{ND}}$,
%description={Nutzungsdauer einer Maschine},
%type=symbolslist,%
%sort=a
%}



% Alternative Definition von Glossareintr\"agen
%
%\newglossaryentry{glos:AD}{
%first=Active Directory\textsuperscript{GL},
%name=Active Directory,
%description={Active Directory ist in einem Windows 2000/Windows
%Server 2003-Netzwerk der Verzeichnisdienst, der die zentrale
%Organisation und Verwaltung aller Netzwerkressourcen erlaubt. Es
%erm\"oglicht den Benutzern \"uber eine einzige zentrale Anmeldung den
%Zugriff auf alle Ressourcen und den Administratoren die zentral
%organisierte Verwaltung, transparent von der Netzwerktopologie und
%den eingesetzten Netzwerkprotokollen. Das daf\"ur ben\"otigte
%Betriebssystem ist entweder Windows 2000 Server oder
%Windows Server 2003, welches auf dem zentralen
%Dom\"anencontroller installiert wird. Dieser h\"alt alle Daten des
%Active Directory vor, wie z.\,B. Benutzernamen und
%Kennw\"orter.\protect\seFootcite{Vgl.}{S. 200}{Dud09}}
%}














