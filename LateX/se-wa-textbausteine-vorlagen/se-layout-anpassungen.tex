\section{Anpassungen des Gesamtlayouts}

\subsection{\"Anderung des vertikalen Zwischenraums beim Start eines neuen Kapitels}

Um den vertikalen Zwischenraum zu ver\"andern, den \LaTeX{} automatisch beim 
Start eines neuen Kapitels erzeugt, kann das Kommando \verb+\seNoChapterSkip+ 
verwendet werden. Dieses Kommando wird direkt vor \verb+\begin{document}+ eingef\"ugt.
Es besitzt einen optionalen Parameter, \"uber den ein Wert angegeben werden kann. Der 
Defaultwert ist \texttt{-14mm}. Damit wird erreicht, dass bei einem neuen Kapitel kein 
zus\"atzlicher vertikaler Zwischenraum eingef\"ugt wird. 

Beispiele:

\begin{seList}
\item
\verb+\seNoChapterSkip{}+\newline \verb+\begin{document}+ \newline
Es wird kein vertikaler Zwischenraum beim Beginn eines neuen Kapitels erzeugt.
\item
\verb+\seNoChapterSkip[11.5mm]+\newline \verb+\begin{document}+ \newline
Es wird der vertikale Zwischenraum erzeugt, der auch ohne Angabe dieses 
Kommandos Verwendung findet.
\item
\verb+\seNoChapterSkip[21.5mm]+\newline \verb+\begin{document}+ \newline
Im Vergleich zu dem standardm\"a{\ss}ig erzeugten vertikalen Zwischenraum 
wird ein 10\,mm gr\"o{\ss}erer Zwischenraum beim Start eines neuen Kapitels 
erzeugt.
\end{seList}

\subsection{M\"ogliche Layout-\"Anderungen f\"ur Seminararbeiten}

\subsubsection{Verwendung kleinerer Schriftgr\"o{\ss}en f\"ur \"Uberschriften}

Die Verwendung kleinerer Schriftgr\"o{\ss}en f\"ur \"Uberschriften wird durch 
die Angabe des Kommandos \verb+\KOMAoption{headings}{small}+ direkt 
vor \verb+\begin{document}+ erreicht.

Soll dieses Kommando bei Seminarbeiten mit \verb+\seNoChapterSkip{}+ kombiniert 
werden, ist die folgende Reihenfolge erforderlich:

\begin{seList}
\item[] \verb+\KOMAoption{headings}{small}+\newline
\verb+\seNoChapterSkip[-12.25mm]+\newline
\verb+\begin{document}+
\end{seList}

Da kleinere Schriftgr\"o{\ss}en f\"ur die \"Uberschriften verwendet werden, sollte das Kommando \verb+\seNoChapterSkip+ mit 
dem optionalen Parameter \texttt{-12.25mm} aufgerufen werden.

\subsubsection{Unterdr\"uckung des Seitenvorschubs f\"ur die folgenden Kapitel}

Das Kommando \verb+\seChaptersWithoutNewpage{}+ unterdr\"uckt den Seitenvorschub des \verb+\chapter+-Kom\-man\-dos f\"ur die folgenden Kapitel. 

Wenn dieses Kommando in Kombination mit \verb+\seNoChapterSkip{}+ benutzt wird, dann sollte nach jedem Kapitelende noch das Kommando 
\verb+\seChapterEndSkip{}+ ausgef\"uhrt werden, damit ein vern\"unftiger Abstand zur folgenden Kapitel\"uberschrift entsteht.

\verb+\seChapterNewpage{}+ erzeugt f\"ur die folgenden Kapitel wieder Seitenvorsch\"ube.

