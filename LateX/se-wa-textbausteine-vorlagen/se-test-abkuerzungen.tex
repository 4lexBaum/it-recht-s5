\newcommand{\dateiAbk}{\texttt{wa-abkuerzungen.tex}}
%
% Ein kleiner Text, um Abk\"urzungen, Symbole und Glossareintr\"age zu testen
%
%
\section{Kommandos f\"ur die Erzeugung von Abk\"urzungen, Symbolen und Glos\-sar\-eint\-r\"a\-gen}

\subsection{Definition von Abk\"urzungen, Symbolen und Glossareintr\"agen}

Um eine einheitliche Darstellung von Abk\"urzungen, Symbolen und Glossareintr\"agen zu erreichen, 
werden vier neue Kommandos zur Verf\"ugung gestellt:

\begin{seList}
\item 
\verb+\seNewAcronymEntry+\newline
Definition einer neuen Abk\"urzung.
\item 
\verb+\seNewSymbolEntry+\newline
Definition eines neuen Symbols.
\item
\verb+\seNewGlossaryEntry+\newline
Definition eines neuen Eintrags im Glossar.
\item
\verb+\seNewAcronymGlossaryEntry+\newline
Definition eines neuen Eintrags im Glossar, wobei zus\"atzlich eine 
Abk\"urzung definiert wird, die dann auch in das Abk\"urzungsverzeichnis aufgenommen wird.
\end{seList}

Der Datei \dateiAbk{} lassen sich die zugeh\"origen \textbf{Pa\-ra\-me\-ter\-be\-schrei\-bun\-gen}  
entnehmen.
In dieser Datei sind auch Beispiele enthalten, wie Abk\"urzungen, Symbole und Glossareintr\"age mit den 
Standardkommandos definiert werden k\"onnen, was jedoch nicht empfohlen wird!

\subsection{Verwendung von Abk\"urzungen, Symbolen und Glossareintr\"agen im Text}

Innerhalb des Textes wird f\"ur Abk\"urzungen, Symbole und Glossareintr\"age das Kommando \verb+\gls{par1}+ 
verwendet.
\texttt{par1} stellt einen Schl\"ussel dar, der die entsprechende Definition identifiziert (vgl. den Inhalt der Datei
\dateiAbk{}). 

Mit dem Kommando \verb+\glspl+ ist es m\"oglich, beim
Auftreten eines Begriffes,  f\"ur den ein Glossareintrag existiert, bzw.\ beim (ersten) Auftreten einer 
Abk\"urzung f\"ur die Vollform die  \textbf{Pluralform} auszugeben.%
\footnote{Genauer gesagt wird derjenige Wert ausgegeben, der in den 
Kommandos \texttt{\textbackslash{}seNewAcronymEntry}, \texttt{\textbackslash{}seNewGlossaryEntry} bzw. \texttt{\textbackslash{}seNewAcronymGlossaryEntry}
als Pluralform definiert wurde. Die Pluralform k\"onnte man alternativ verwenden, um beispielsweise eine 
Genitivform zu definieren.}

Bei den Kommandos 
\begin{seList}
\item\verb+\seNewAcronymEntry+ und 
\item\verb+\seNewAcronymGlossaryEntry+
\end{seList}
kann durch die Verwendung des optionalen Parameters 
zu\-s\"atz\-lich eine Pluralform f\"ur die Abk\"urzung definiert werden (vgl. \dateiAbk{}).

\subsection{Anwendungsbeispiele}

\subsubsection{Abk\"urzungen}

Die dreimalige Anwendung von \verb+\gls{usb}+ liefert:

\begin{seList}
\item \gls{usb}
\item \gls{usb}
\item \gls{usb}
\end{seList}

Die Anwendung von \verb+\glspl{dm}+ \verb+\glspl{dm}+ \verb+\gls{dm}+ liefert:

\begin{seList}
\item \glspl{dm}
\item \gls{dm}
\item \gls{dm}
\end{seList}

Und auch die \gls{dhbw} soll noch erw\"ahnt werden, um das Abk\"urzungsverzeichnis ein wenig zu f\"ullen.

\subsubsection{Symbole}

Bei einem Symbol wird -- im Gegensatz zu Abk\"urzungen -- beim ersten Auftreten im Text nicht die 
zugeh\"orige Definition ausgegeben. Diese ist aber im Symbolverzeichnis zu finden.

Die zweimalige Anwendung von \verb+\gls{pi}+ liefert:

\begin{seList}
\item \gls{pi}
\item \gls{pi}
\end{seList}

Und jetzt kommt noch ein zweites Symbol f\"ur das Symbolverzeichnis: \gls{ND}

\subsubsection{Glossareintr\"age}

Bei einem Glossareintrag wird beim ersten Auftreten des Begriffes im Text dieser mit \textsuperscript{GL} markiert.
Im Glossar sind die Seitenzahlen angegeben, auf denen der Begriff verwendet wurde. 

Die dreimalige Anwendung von \verb+\gls{glos:AD}+ liefert:

\begin{seList}
\item \gls{glos:AD}
\item \gls{glos:AD}
\item \gls{glos:AD}
\end{seList}

Und hier kommt noch ein Beispiel f\"ur einen Glossareintrag, f\"ur den beim ersten und dritten Auftreten die Pluralform verwendet 
wird:  \verb+\glspl{glos:bs}+  \verb+\gls{glos:bs}+  \verb+\glspl{glos:bs}+ 

\begin{seList}
\item \glspl{glos:bs}
\item \gls{glos:bs}
\item \glspl{glos:bs}
\end{seList}

\subsubsection{Glossareintrag mit einem zus\"atzlichen Eintrag im Ab\-k\"ur\-zungs\-ver\-zeich\-nis}

Nach der ersten Anwendung des Begriffes, f\"ur den ein Glossareintrag erzeugt wurde, wird in der Folge 
jeweils nur noch die Abk\"urzung benutzt. 

Die Kommandoausf\"uhrungen \verb+\glspl{glos:ma}+ \verb+\gls{glos:ma}+  \verb+\glspl{glos:ma}+ haben als 
Ergebnis:

\begin{seList}
\item \glspl{glos:ma}
\item \gls{glos:ma}
\item \glspl{glos:ma}
\end{seList}

\newpage
Und jetzt wird auf einer neuen Seite nochmals \verb+\gls{glos:ma}+ verwendet, um im Glossar die neu hinzugekommene 
Seitennummer zu demonstrieren: \gls{glos:ma}

\subsubsection{Pluralform von Abk\"urzungen}

\textbf{\textsf{Definition einer Abk\"urzung}}

Der Eintrag wurde wie folgt definiert (vgl. \dateiAbk{}):

\vspace{-\baselineskip}
\begin{verbatim}
   \seNewAcronymEntry[URLs]{url}{URL}{Uniform Resource Locator}%
   {Uniform Resource Locators}
\end{verbatim}
\vspace{-\baselineskip}

Die Kommandoausf\"uhrungen \verb+\glspl{url}+ \verb+\gls{url}+  \verb+\glspl{url}+ haben als 
Ergebnis:

\begin{seList}
\item \glspl{url}
\item \gls{url}
\item \glspl{url}
\end{seList}

\seVsd
\textbf{\textsf{Definition eines Glossareintrags mit zus\"atzlicher Abk\"urzung}}

Der Eintrag wurde wie folgt definiert (vgl. \dateiAbk{}):

\vspace{-\baselineskip}
\begin{verbatim}
   \seNewAcronymGlossaryEntry[TAen]{glos:ta}{TA}{Transaktion}%
   {Transaktionen}%
   {Was eine Transaktion ist, sollten Sie ebenfalls bereits wissen!}
\end{verbatim}
\vspace{-\baselineskip}


Die Kommandoausf\"uhrungen \verb+\glspl{glos:ta}+ \verb+\gls{glos:ta}+  \verb+\glspl{glos:ta}+ haben als 
Ergebnis:

\begin{seList}
\item \glspl{glos:ta}
\item \gls{glos:ta}
\item \glspl{glos:ta}
\end{seList}

%2013-07-08
\subsubsection{Literaturverweise in Glossareinträgen}

Auch bei Glossareinträgen müssen natürlich Literaturverweise angegeben werden. 
Wird eine Literaturquelle erstmalig in einem Glossareintrag verwendet, dann tritt das Problem auf, 
dass sie von BibTeX nicht gefunden wird. Ein \textsl{Workaround} besteht darin, für die entsprechenden 
Literaturverweise \verb+\nocite{key}+-Kommandos anzugeben. \verb+key+ ist hierbei der zugehörige Schlüssel 
des Eintrags in der .bib-Datei.\footnote{Achtung: Ein \texttt{\textbackslash{}nocite}-Kommando sollte nur in absoluten Ausnahmefällen 
eingesetzt werden, da hiermit Einträge im Literaturverzeichnis erzeugt werden können, für die (möglicherweise) kein 
Literaturverweis innerhalb der Arbeit existiert.}

\newpage

 