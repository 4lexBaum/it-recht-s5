\chapter{Kritik}
Die GEMA hat in der Öffentlichkeit einen schwierigen Stand, wobei die Kritik aus unterschiedlichen Perspektiven strömt. Im Folgenden werden dafür die Sichten eines Mitglieds, eines Diskothekenbetreibers sowie die eines Endnutzers kurz dargestellt.

\section{Mitglieder}

Die im \vref{mitlgiederstruktur} beschriebenen Mitgliedsstufen der GEMA führen automatisch zu einer unterschiedlichen Mitbestimmung, wobei nur knapp 5.000 der insgesamt 65.000 Mitglieder ein direktes Wahlrecht bei den Versammlungen haben und 60.000 angeschlossene sowie außerordentliche Mitglieder von gerade mal 64 Delegierten vertreten werden. Bis heute sind über 1.800 \glqq Anti-GEMA-Petitionen\grqq~beim Deutschen Bundestag eingegangen, was jedoch größtenteils durch die Kontroverse der Sperrtafeln angeheizt wurde.\seFootcite{Vgl.}{}{PM:G} Weiterhin ist jedes Mitglied dazu verpflichtet, jedes Werk, welches potentiell in der Öffentlichkeit gespielt werden könnte, anzumelden. Für alle bereits bestehenden sowie zukünftigen Werke besitzt die GEMA nach § 1 des GEMA-Berechtigungsvertrages ausschließliche Nutzungsrechte. Außerdem ist es nicht möglich, einige Werke zur nicht-kommerziellen Nutzung oder ohne Lizenz freizustellen.

\section{Clubbetreiber und Diskothekenbesitzer}
Aus Sicht der Clubbetreiber und Diskothekenbesitzer hagelt es schon seit Jahren Kritik aus ganz Deutschland. 2012 forderte die GEMA unfassbare 1.000 Prozent mehr Geld für die Diskothek-Gebühren -- für Musikveranstaltungen wurde im Gegenzug der Prozentsatz gesenkt. \seFootcite{Vgl.}{}{TS:G} Die GEMA begründete diese Entscheidung mit der zuvor ungleichen Verteilung zwischen Kulturveranstaltungen und Diskotheken, wobei letztere in den vergangen Jahren viel weniger gezahlt hätten. Für die Clubbetreiber kam es jedoch noch schlimmer, denn diese mussten nicht mehr für die Zahl der Anwesenden zahlen, sondern für die maximale Kapazität des Clubs, obwohl dieser an Tagen auch nur zu 50\% ausgelastet sein könnte. Aus dieser Reform entstand die Gegenbewegung des Aktionsbündnis \glqq Kultur-retten.de\grqq, die bis heute laut eigener Aussage mit über 305.000 Unterstützern die größte deutsche Petition initiierte. 

\section{Endnutzer}
Der Endnutzer der künstlerischen Werke muss gerade durch den Rechtsstreit zwischen YouTube und GEMA immer wieder auf einige Videos innerhalb von Deutschland verzichten. Was aus Sicht des Endverbraucher oftmals völlig unverständlich ist, sichert jedoch vielen Künstlern überhaupt ein Einkommen. Weiterhin steigen die GEMA-Gebühren für Diskotheken sowie Musikveranstaltungen, was sich wiederum direkt auf den Endverbraucher auswirkt. Die GEMA hat folglich einen negativ angehauchten Stand in der deutschen Gesellschaft, was sich größtenteils durch fehlendes Wissen über ihre eigentliche Funktion begründen lässt.