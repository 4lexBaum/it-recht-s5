\chapter{Zusammenfassung}
Rückblickend lässt sich zusammenfassen, dass die GEMA als Verwertungsgesellschaft eine lange Entstehungsgeschichte vorzuweisen hat und trotz allgemeiner Kritik eine bis heute wichtige Rolle beim Schutz von Komponisten, Textdichtern und Verlegern einnimmt. Nach der noch zu Beginn der Tantiemenbewegung herrschenden Konkurrenzsituation unter den Verwertungsgesellschaften, folgte durch die staatliche Monopolisierung von privaten Einrichtungen innerhalb des Dritten Reiches auch eine Monopolisierung der Verwertunggesellschaften in Deutschland unter der Stagma - die bis heute anhält.

Die Existenz der GEMA begründet sich auf dem Urheberrechtsgesetz sowie dem Urheberrechtswahrnehmungsgesetz, wodurch die Verwertungsgesellschaft berechtigt ist, die Nutzungsrechte ihrer Mitglieder zu verwalten und damit Gebühren für die lizenzpflichtigen Werke zu erheben. 

Der seit Jahren andauernde Rechtsstreit zwischen YouTube und der GEMA lässt sich in drei Klagen unterteilen. Im ersten Fall der Unterlassungsklage -- der immer noch vor Gericht ist -- wurde YouTube nur als Störer verurteilt, wodurch der Streaming-Dienst veröffentlichte Inhalte auf Lizenzverletzungen prüfen musste. Da zwischen den beiden Parteien nach dem einzigen Vertrag zwischen 2007 und 2009 kein neuer mehr zustande kam, verklagte die GEMA YouTube auf Schadensersatz, da keine Nutzungsgebühren mehr entrichtet wurden. Das Gericht urteilte zugunsten der Google-Tochter, begründet wurde dies durch die Tatsache, dass YouTube nicht selbst, sonder die Nutzer die Inhalte hochladen und somit für die Lizenzpflicht verantwortlich sind, obwohl das Geschäftsmodell auf den werbefinanzierten Videos beruht -- das Urteil liegt zurzeit beim Bundesgerichtshof. Aus den Streitigkeiten ergab sich der medial wohl bekannteste Rechtsstreit um die sogenannten von YouTube geschalteten Sperrtafeln. Der Internetdienst ließ seit Mitte 2011 Sperrtafeln für Werke schalten, die gegen das Nutzungsrecht verstoßen und teilte seinen Nutzer mit, dass der Grund dafür alleinig bei der GEMA liegen würde. Daraufhin ging die Verwertungsgesellschaft vor Gericht und gewann den Prozess der irreführenden Sperrtafeln, wodurch YouTube mittlerweile eine neutral ausgerichtete Nachricht anzeigen lässt. 

Kritik an der GEMA gibt es von den unterschiedlichsten Seiten, sowohl von den Mitgliedern, als auch von den Musiknutzern, angefangen von Diskothekenbetreibern bis hin zu privaten Endverbrauchern. Letztere Kritik gründet sich auf die von YouTube geschalteten Sperrtafeln, die den öffentlichen Druck auf die GEMA enorm erhöht hat.
\newpage
Mit Blick auf die nächsten Jahre könnte es gerade im Streit zwischen YouTube und der GEMA zu einem Ende kommen, da durch die Ablösung des Urheberrechtswahrnehmungsgesetzes durch das Verwertungsgesellschaftengesetz eine neue, europaweite Regelung für Verwertungsgesellschaften geschaffen wurde, sodass der Internet-Dienst regionsübergreifend Lizenzgebühren entrichten könnte.